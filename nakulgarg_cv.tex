\documentclass[10pt]{article}
\usepackage{hyperref}
\usepackage{calc}
\usepackage{comment}
\usepackage{url}

\makeatletter
\newlength{\bibhang}
\setlength{\bibhang}{1em} %1em}
\newlength{\bibsep}
 {\@listi \global\bibsep\itemsep \global\advance\bibsep by\parsep}
\newenvironment{bibsection}%
        {\begin{enumerate}{}{%
%        {\begin{list}{}{%
       \setlength{\leftmargin}{\bibhang}%
       \setlength{\itemindent}{-\leftmargin}%
       \setlength{\itemsep}{\bibsep}%
       \setlength{\parsep}{\z@}%
        \setlength{\partopsep}{0pt}%
        \setlength{\topsep}{0pt}}}
        {\end{enumerate}\vspace{-.6\baselineskip}}
%        {\end{list}\vspace{-.6\baselineskip}}
\makeatother
\reversemarginpar

\usepackage[paper=letterpaper,
            %includefoot, % Uncomment to put page number above margin
            marginparwidth=1.2in,     % Length of section titles
            marginparsep=.05in,       % Space between titles and text
            margin=1in,               % 1 inch margins
            includemp]{geometry}

\setlength{\parindent}{0in}

\usepackage[shortlabels]{enumitem}

\usepackage{fancyhdr,lastpage}
\pagestyle{fancy}
%\pagestyle{empty}      % Uncomment this to get rid of page numbers
\fancyhf{}\renewcommand{\headrulewidth}{0pt}
\fancyfootoffset{\marginparsep+\marginparwidth}
\newlength{\footpageshift}
\setlength{\footpageshift}
          {0.5\textwidth+0.5\marginparsep+0.5\marginparwidth-2in}
\lfoot{\hspace{\footpageshift}%
       \parbox{4in}{\, \hfill %
                    \arabic{page} of \protect\pageref*{LastPage} % +LP
%                    \arabic{page}                               % -LP
                    \hfill \,}}

% Finally, give us PDF bookmarks
\usepackage{color,hyperref}
\definecolor{darkblue}{rgb}{0.0,0.0,0.3}
\hypersetup{colorlinks,breaklinks,
            linkcolor=darkblue,urlcolor=darkblue,
            anchorcolor=darkblue,citecolor=darkblue}

%%%%%%%%%%%%%%%%%%%%%%%% End Document Setup %%%%%%%%%%%%%%%%%%%%%%%%%%%%


%%%%%%%%%%%%%%%%%%%%%%%%%%% Helper Commands %%%%%%%%%%%%%%%%%%%%%%%%%%%%

\newcommand{\makeheading}[2][]%
        {\hspace*{-\marginparsep minus \marginparwidth}%
         \begin{minipage}[t]{\textwidth+\marginparwidth+\marginparsep}%
             {\large \bfseries #2 \hfill #1}\\[-0.15\baselineskip]%
                 \rule{\columnwidth}{1pt}%
         \end{minipage}}

\renewcommand{\section}[1]{\pagebreak[3]%
    \hyphenpenalty=10000%
    \vspace{1.3\baselineskip}%
    \phantomsection\addcontentsline{toc}{section}{#1}%
    \noindent\llap{\scshape\smash{\parbox[t]{\marginparwidth}{\raggedright #1}}}%
    \vspace{-\baselineskip}\par}

\newenvironment{outerlist}[1][\enskip\textbullet]%
        {\begin{itemize}[#1,leftmargin=*]}{\end{itemize}%
         \vspace{-.6\baselineskip}}

\newenvironment{lonelist}[1][\enskip\textbullet]%
        {\begin{list}{#1}{%
        \setlength{\partopsep}{0pt}%
        \setlength{\topsep}{0pt}}}
        {\end{list}\vspace{-.6\baselineskip}}

\newenvironment{innerlist}[1][\enskip\textbullet]%
        {\begin{itemize}[#1,leftmargin=*,parsep=0pt,itemsep=0pt,topsep=0pt,partopsep=0pt]}
        {\end{itemize}}

\newenvironment{loneinnerlist}[1][\enskip\textbullet]%
        {\begin{itemize}[#1,leftmargin=*,parsep=0pt,itemsep=0pt,topsep=0pt,partopsep=0pt]}
        {\end{itemize}\vspace{-.6\baselineskip}}

\newcommand{\blankline}{\quad\pagebreak[3]}
\newcommand{\halfblankline}{\quad\vspace{-0.5\baselineskip}\pagebreak[3]}

\newcommand\doilink[1]{\href{http://dx.doi.org/#1}{#1}}
\newcommand\doi[1]{doi:\doilink{#1}}

%\providecommand*\url[1]{\href{#1}{#1}}
%\renewcommand*\url[1]{\href{#1}{\texttt{#1}}}
\providecommand*\email[1]{\href{mailto:#1}{#1}}

\providecommand\BibTeX{{B\kern-.05em{\sc i\kern-.025em b}\kern-.08em
    \TeX}}
\providecommand\Matlab{\textsc{Matlab}}

%%%%%%%%%%%%%%%%%%%%%%%%% Begin CV Document %%%%%%%%%%%%%%%%%%%%%%%%%%%%
\begin{document}
\makeheading{Nakul Garg}

\section{\textbf{Contact Information}}

\newlength{\rcollength}\setlength{\rcollength}{1.4in}%


3245 Brendan Iribe Center \hfill {+1 202-725-1919} \newline
8125 Campus Drive \hfill {\email{nakul@cs.umd.edu}} \newline
%College Park, MD 20742 \hfill \href{https://www.cs.umd.edu/~nakul}{https://www.cs.umd.edu/~nakul/} \newline
College Park, MD 20742 \hfill \url{https://www.cs.umd.edu/~nakul} \newline

\section{\textbf{Research Interests}}
Mobile Computing, Sensing, Wireless Networking\\

\section{\textbf{Education}}
\textbf{Ph.D.}, Computer Science \hfill{Aug. 2019--Present}\\
University of Maryland, College Park\\

\textbf{Bachelor of Technology}, ECE \hfill{Aug. 2014--May 2018}\\
Guru Gobind Singh Indraprastha University\\

%\textbf{St. Marks School},
%New Delhi, India \\
%12\textsuperscript{th} C.B.S.E, 90\%\hfill{Jul. 2014}\\
%10\textsuperscript{th} C.B.S.E, 82\%\hfill{Jul. 2012}

\section{\textbf{Publications}}
Enabling Self-defense in Small Drones\\
\textbf{Nakul Garg}, Nirupam Roy\\
ACM HotMobile 2020\\

Poster: Acoustic Sensing for Detecting Projectile Attacks on Small Drones\\
\textbf{Nakul Garg}, Nirupam Roy\\
ACM HotMobile 2020\\

Poster: DRIZY- Collaborative Driver Assistance Over Wireless Networks\\
\textbf{Nakul Garg}, Ishani Janveja, Divyansh Malhotra, Chetan Chawla, Pulkit Gupta, Harshil Bansal, Aakanksha Chowdhery, Prerana Mukherjee, Brejesh Lall\\
ACM MobiCom 2017\\
%\textbf{N. Garg}, I. Janveja, D. Malhotra, C. Chawla, P. Gupta, H. Bansal, A. Chowdhery, P. Mukherjee, and Brejesh Lall. ``DRIZY---Collaborative Driver Assistance Over Wireless Networks," \textbf{ACM MobiCom} Poster, Utah, USA, 2017.

\section{\textbf{Experience}}
\textbf{Research Intern} \hfill {Sept. 2018---July 2019}\\
Computer Science Department, IIT Delhi\\
Advisor : Professor Rijurekha Sen\\
\textbf{Technical Advisor} \hfill {May 2018---Sept. 2018}\\
Celestini Project India, Marconi Society, Google, IIT Delhi\\
Advisor : Dr. Aakanksha Chowdhery and Professor Brejesh Lall\\
\textbf{Research Intern} \hfill {Jun. 2017---Sept. 2017}\\
Celestini Project India, Marconi Society, Google, IIT Delhi\\
Advisor : Dr. Aakanksha Chowdhery and Professor Brejesh Lall\\

\section{\textbf{Teaching}}
CMSC420 Advanced Data Structures, Fall 2019\\
Instructor: Prof. Jason Filippou\\

CMSC417 Computer Networks, Spring 2020\\
Instructor: Prof. Nirupam Roy\\\\\\\\

%\section{Positions of Responsibility}
%\textbf{Chairperson} BVP IEEE Student Branch \hfill {2017 - 18}
%\begin{innerlist}
%\item Led a team of 200 students to organize one of the biggest technical hackathon in Delhi, India.
%\end{innerlist}
%\textbf{Vice--Chair} of Robotics Society, BVCOE \hfill {2016 - 17}
%\begin{innerlist}
%\item Conducted workshops across India to teach programming concepts on Arduino/Raspberry, Image Processing and Project Management to 200+ students.
%\end{innerlist}
%\textbf{Head Event--Manager}, Fervour - Technical Fest of college. \hfill {2017}
%\begin{innerlist}
%\item Organized 17 technical events in campus with an outreach of 1000+ students in India.
%\end{innerlist}
\section{\textbf{Awards and Scholarships}}

ACM HotMobile 2020 Student Travel Award \hfill 2020\\

University of Maryland Graduate School Dean's Fellowship \hfill 2019\\

Outstanding Student Volunteer Award by IEEE Delhi Section. \hfill 2018\\

%\item First position at \textbf{EV Hackathon}, India - Australia joint initiative. \hfill 2018
Winner of World Food India Hackathon. Awarded by President of India. \hfill 2017\\

Winner of Celestini Project India 2017,IIT Delhi \& Marconi Society. \hfill 2017\\

ACM MobiCom 2017 Student Travel Award \hfill 2017\\

%\item Travel grants: NSF for Mobicom 2017, ACM SIGMOBILE \hfill 2017

1\textsuperscript{st} in eYantra - National Robotics Competition, IIT Bombay \hfill 2017\\

2\textsuperscript{nd} in eYantra - National Robotics Competition, IIT Bombay \hfill 2016\\

%\item Ranked 44 in IEEE XTreme Hackathon\hfill  2015

%\item 1\textsuperscript{st} in Robotron TechMarathon, DDUC, Delhi University\hfill 2014

Among top 6 in National CBSE Science Exhibition \hfill  2013\\

%\item 4\textsuperscript{th} in International Quanta, CMS School, Lucknow\hfill 2013

1\textsuperscript{st} in Regional level CBSE Science Exhibition \hfill  2012\\

1\textsuperscript{st} in Annual School Science Exhibition\hfill 2011\\

1\textsuperscript{st} in Annual School Science Exhibition\hfill  2010\\

2\textsuperscript{nd} in Annual School Science Exhibition\hfill  2009\\

%\section{Technical Projects}
%\begin{innerlist}
%    \item \textbf{"Aerogram"} Particualte Matter Sensor Network in 200 Buses across Delhi. \hfill {2019}
%    \item \textbf{"Near Sensor ML"} Reducing latency by bringing ML at edge. \hfill {2018}
%    \item \textbf{"PlugFree"} Autonomous charging of Drones. (Sponsored by GSU, USA) \hfill {2018}
%    \item \textbf{"Augur"} Visible Light Communication using Smartphone Camera \hfill {2018}
%    \item \textbf{"DRIZY"} Collaborative Driver Assistance Over Wireless Networks \hfill {2017}
%    \item \textbf{Li-Fi} (Data transfer through light) Demonstration \hfill {2017}
%    \item \textbf{"PUSHPAK"} Aerial Surveillance Quadcopter with Rover\hfill {2016}
%    \item \textbf{Touch--Screen} Based Home Automation \hfill {2016}
%    \item \textbf{IOT} based Temperature Logger with Remote Access\hfill {2016}
%    \item \textbf{FireBird V Robot} -- Mars Rover Navigation and 3D Modelling\hfill {2016}
%    \item Raspberry pi-3 based Personal \textbf{Cloud Storage} \hfill {2015}
%    \item Anti Car Theft System with SMS alert application\hfill {2014}
%    \item \textbf{Zig--Bee} based Swarm Robotics\hfill {2014}
%    \item Automatic Rubik’s Cube Solver\hfill {2014}
%    \item \textbf{Wireless Odometer}\hfill {2012}
%\end{innerlist}

%\section{Skills and \\Tools}
%Languages : C/C++, Python, MATLAB, OpenCV, \LaTeX, Verilog\\
%Technologies : Arduino, Raspberry Pi, Deep Learning, Computer Vision
%
%\section{Society Memberships}
%Institute of Electrical and Electronics Engineers (IEEE)\\
%Association for Computing Machinery (ACM)

\end{document}